% !TEX encoding = UTF-8 Unicode
%%==================================================
%% thanks.tex for SJTU Bachelor Thesis
%% version: 0.5.2
%% Encoding: UTF-8
%%==================================================

\chapter{Conslusion}
\label{chap:conclusion}

In this chapter we summarize all contents above of this thesis and 
draw some conclusions on the stock return series prediction work we have done.

Up to now,
we have formally introduced the hidden Markov model (HMM) in Ch.\,\ref{chap:HMM},
including the model formulation and important statistics, 
model estimation methods, and ways to perform forecasts and decoding.
As a simple dynamic Bayesian network,
HMM is easy to be realized, 
and it excavates the information not only about the stock returns themselves 
but also the classifications of them.
which enables us to analyze the data from the angles of points and also intervals.
The number of model parameters in under control of the number of hidden states,
at the level of $\co(N^2)$ ($N$ represents the number of states),
which is usually small in empirical analysis.
From these perspectives, 
HMM is quite suitable to apply for stock return series analysis and prediction,
due to its easy implementation, appropriate complexity and small number of parameters.

In Ch.\,\ref{chap:system} we have constructed the stock return series prediction system,
which combines model initialization with K-Means,
model estimation, historical analysis (with HMM global decoding) and visualization.
The system is adaptive to different data populations 
(w.r.t.\,target index, time horizon, observation frequency, etc.) and 
exogenous parameters (e.g.\,the number of hidden states).
It is also complete, encapsulated and user-friendly 
(see Appendix \ref{app:code} for the user manuscript and source codes).

In Ch.\,\ref{chap:positive} we have carried out empirical analysis on 
both U.S. (the S\&P 500 Index) and Chinese (the CSI300 Index) stock markets.
With some modest assumptions,
the system has been well functioning for all the analysis.
From the perspective of prediction correctness,
the system is effective and always has a win ratio greater than 50\%,
and sometimes approaches 60\% under certain circumstances.
We can also draw the conclusion that data populations with higher observation frequencies
tend to outperform those with lower frequencies in this system
due to the difference in the amount of information contained in the data.
From the view of global decoding,
we come up to the conclusion that data populations with longer time horizon 
(i.e.\,longer observation period)
tend to have (better) global decoding results that more fit our usual acknowledgements.
Potential for improvement of the HMM-based system lies in 
the incorporation of resampling or sample reweighting,
which shall largely ameliorate the time lag effect.
This part will be briefly introduced in Ch.\,\ref{chap:future}.

To sum up,
we have successfully constructed the HMM-based stock return series prediction system
and have accomplished all the aims of this thesis.
We want to reinstate that this thesis in aimed to fulfill a complete implementation
of stock return prediction with HMM,
while we have no intention to realize the improvements of model-level techniques.
Finally note that the model is imperfect and there exist many unaddressed issues.
We cover some of them in the last chapter and hope to incorporate such improvements in future works.