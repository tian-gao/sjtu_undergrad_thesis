%%==================================================
%% bigabstract.tex for SJTU Bachelor Thesis
%% version: 0.5.2
%% Encoding: UTF-8
%%==================================================

\begin{bigabstract}

In the area of quantitative finance,
research and analysis on stock return series is one of the most concentrated topics 
shared by both the industry and the academia.
Conditional on different market states,
the distributions of stock returns vary largely,
and the hidden Markov model is able to quantitatively depict the state-dependency of stock returns.

This thesis is intended to implement the hidden Markov model to 
analyze and describe the hidden states of stock markets,
from which to deduce the conditional distributions of the returns,
and carry out stock return series prediction based on the model.

In this thesis, 
we provide analyses and explanations with details on the model.
Formal definition and formulation of a hidden Markov is provided,
followed by the derivation of marginal distributions and other primary statistics.
We introduce the forward and backward procedure and 
expectation maximization algorithm for model parameters estimation,
after which forecast distributions are also covered.
In order to excavate information about the historical hidden states,
the concepts of state probabilities, local decoding and global decoding are introduced,
and we use the Viterbi algorithm to conduct global decoding for the hidden Markov model.

Based on this theoretical model and some other related knowledge and techniques,
we construct a complete stock return series prediction system.
We achieve the aim of prediction through three steps,
model initialization with K-Means clustering,
model estimation with expectation maximization algorithm,
and iterative one-step forecasts for the out-of-sample data.
The task of model realization is accomplished through Python programming.

We then use the realized system to perform empirical analysis and model validation 
on the CSI 300 Index from Chinese stock market and the S\&P 500 Index from the U.S. stock market.
After making appropriate assumptions and conjectures,
we in order analyze the S\&P 500 daily return data, CSI 300 daily data,
and CSI 300 medium frequency (60min and 10min) data.
Thorough analyses and comparions are carried out and numerical results are provided,
along with presentations of results visualization.

Additionally, we provide further discussions on unaddressed issues remained 
in the model construction part and empirical analysis part,
including the time lag effect and model estimations for other more general hidden Markov models.
To ameliorate the time lag effect we propose two potential solutions,
which are adopting rolling windows for sample data censoring and 
a data reweighting algorithm named exponantially weighted expectation maximization.
For more general hidden Markov models,
we restate the problem formulation and provide the derivations of some key statistics.
The particle filter method is also introduced as a standard tool for model estimation.
We hope to solve these remaining problems and further our research on the model in future works.

\end{bigabstract}
