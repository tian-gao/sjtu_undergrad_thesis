% !TEX encoding = UTF-8 Unicode
%%==================================================
%% thanks.tex for SJTU Bachelor Thesis
%% version: 0.5.2
%% Encoding: UTF-8
%%==================================================

\chapter{Introduction}
\label{chap:introduction}
We introduce the brief history and status quo of stock return series analysis and prediction
in the very first chapter of this thesis.
The topic has received heated discussion since the invention of exchange traded stocks.
In Sec.\,\ref{sec:introduction:timeseries} and \ref{sec:introduction:technique}
we briefly introduce the elementary concepts about financial time series and stock returns,
and some mature and standard techniques for time series analysis.
In Sec.\,\ref{sec:introduction:background} we introduce the novel hidden Markov model,
and explain the background and aims of this thesis.
Then in Sec.\,\ref{sec:introduction:organization} we show the framework of the thesis,
and list the notations, abbreviations and jargons that are used in the following chapters
for the reader's convenience.

%%%%%%%%%%%%%%%%%%%%%%%%%%

\section{Financial time series and stock returns}
\label{sec:introduction:timeseries}
A time series, by definition, 
is a sequence of data points that are recorded with a timeline.
The observations can be made either over a continuous interval or at discrete time nodes;
either regularly spaced or irregularly spaced.
A financial time series is a special kind of time series of which
observations are prices of financial assets like bonds, stocks or stock indices. 

Of all kinds of financial markets,
stock market is the most volatile and popular one.
Analysis on stocks has been carried out since the very first day 
when exchange traded stocks were invented.
In this work we focus our research on the stock markets;
more specifically, 
the CSI 300 index in the Chinese market and the S\&P 500 index in the U.S. market,
which are weighted average of the most 300/500 stocks in the corresponding markets 
and considered to be the best indicators of the markets.

One of the most important and fundamental indicators of 
evaluating the performance of a financial asset is its return.
Mathematically, the return $r$ of a stock at time $t$ is defined as
		\begin{equation}
		\label{eq:return}
		r_t = \frac{P_{t} - P_{t-1}}{P_{t-1}} = \frac{P_{t}}{P_{t-1}} - 1,
		\end{equation}
where $P_t$ stands for the price of the stock at time $t$.
Eq.\,\ref{eq:return} is also known as the simple return of a stock.
For a $k$-period financial time series,
the return at the end of the period is calculated cumulatively by the returns during the period
		\begin{equation}
		\label{eq:return_k}
		\begin{aligned}
		r_t(k) & = \frac{P_{t} - P_{t-k}}{P_{t-k}} \\
		& = (1 + r_{t-k+1})(1 + r_{t-k+2})\cdots(1 + r_{t}) - 1 \\
		& = \prod_{i=t-k+1}^{t}(1 + r_i) - 1.
		\end{aligned}
		\end{equation}
In order to make the computations more easy and elegant,
the log return has been introduced and the equations above can be transformed into the follows:
		\begin{subequations}
		\begin{align}
		\label{eq:logreturn}
		R_t & = \log\frac{P_{t}}{P_{t-1}}, \\ 
		\label{eq:logreturn_k}
		R_t[k] & = \log\frac{P_{t}}{P_{t-k}} = 
			\log\frac{P_{t}}{P_{t-1}}\cdot\frac{P_{t-1}}{P_{t-2}}
			\cdot\cdots\cdot\frac{P_{t-k+1}}{P_{t-k}} = 
			\sum_{i=t-k}^{t} R_i.
		\end{align}
		\end{subequations}
Eq.\,\ref{eq:logreturn} is the definition of log return and 
Eq.\,\ref{eq:logreturn_k} implements the additivity of logarithm.

%%%%%%%%%%%%%%%%%%%%%%%%%%

\section{Traditional time series analysis techniques}
\label{sec:introduction:technique}
Time series analysis can be dated far back to early the last century,
for different goals in different research areas.

In terms of researched domain, 
time series analysis can be divided into frequeny-domain analysis and time-domain analysis;
the former includes spectral analysis and wavelet analysis, etc., 
while the latter is composed mainly of correlation analyses.
In terms of statistical method,
the models are further classified as parametric, semi-parametric and non-parametric.
With respect to the observed variable(s) the models can be divided into 
linear and nonlinear, univariate and multivariate. 

In 1951, Peter Whittle introduced the autoregressive–moving-average (ARMA) model in \cite{Whittle:1951hy},
and the method is popularized by George E.\,P.~Box and Gwilym Jenkins.
Later the model was developed to consider integrated time series and derived 
the autoregressive integrated moving average (ARIMA) model and 
autoregressive fractionally integrated moving average (ARFIMA) model.
Extension of the models to deal with multi-variables, 
i.e.\,vector-valued variables,
is then called the vector autoregression (VAR) model.
Models related to heteroskedasticity of the time series were also developed,
the most famous two being autoregressive conditional heteroskedasticity (ARCH) model
and generalized autoregressive conditional heteroskedasticity (GARCH) model.

With development of Bayesian statistics,
a kind of models named dynamic Bayesian network has been built for time series analysis,
of which the simplest one is called hidden Markov model (HMM).
HMM is a statistical Markov model,
where there is an observed time series and 
an underlying series that is assumed to be an Markov chain with unobserved (hidden/latent) states.
This thesis is concentrated on the very model.

%%%%%%%%%%%%%%%%%%%%%%%%%%

\section{Background and aims}
\label{sec:introduction:background}
In 1982, a private hedge fund investment company named Renaissance Technologies 
was founded in New York by Dr.\,James H. Simons,
who is also a mathematician well-known in the area of pattern recognition.
Employment of plenty of mathematicians and statisticians and
widely applications of statistical models to quantitative finance,
these elements help Renaissance become one of the most successful investment companies.
The most famous fund of the company is the Medallion fund,
and word is that the core model implemented in Medallion's quantitative investment strategies is the HMM.
With great interest in the model itself and admiration to Dr.\,Simons,
the author decided to study the HMM and try to figure out the secret to the success of Renaissance.

Despite that it is almost impossible to reproduce the investment strategies based on HMM,
we can perform thorough financial time series analysis with the model,
and construct a complete stock return series prediction system on top of the analysis results.
Therefore, we aim to build such a system,
in which we shall apply the hidden Markov model.
We hope to achieve this goal by three sub-aims:
		\begin{itemize}
		\item \textbf{Aim 1.} \\
		Construct the entire system for stock return series prediction. 
		The system should include data pre-processing, model construction and estimation, 
		and eventually predictions with detailed analysis.
		\item \textbf{Aim 2.} \\
		Estimate the HMM parameters in order to calibrate the model with accessible market data.
		\item \textbf{Aim 3.} \\
		Carry out empirical analysis on both Chinese and U.S. stock markets. 
		Verify the effectiveness of the prediction system and provide detailed analysis and comparisons.
		\end{itemize}
Finally, we hope to realize the system with Python programming with
each module encapsulated and flexible for different data and parameters.

%%%%%%%%%%%%%%%%%%%%%%%%%%

\section{Thesis organization}
\label{sec:introduction:organization}
This thesis is composed of seven chapters and two appendices.

In Ch.\,\ref{chap:preliminary} we introduce some prerequisite
knowledge and models that will be later used in HMM introduction or system construction.
Ch.\,\ref{chap:HMM} formally introduce the hidden Markov model,
including model formulation, some key concepts and related algorithms.
Then with knowledge of the model, 
we show construction of the full stock return series prediction system in Ch.\,\ref{chap:system},
and then conduct empirical analysis with the system in Ch.\,\ref{chap:positive},
on both Chinese CSI 300 index and U.S. S\&P 500 index.
The Python codes for model realization are provided in Appendix \ref{app:code} 
along with a short user manuscript.
Part of the visualization results that are not presented in Ch.\,\ref{chap:positive} 
are then listed in Appendix \ref{app:fig}.
Ch.\,\ref{chap:conclusion} concludes the thesis,
and in Ch.\,\ref{chap:future} we make brief discussions on some remaining issues 
and thoughts for future works.

%%%%%%%%%%%%%%%%%%%%%%%%%%

\section{Notations, abbreviations and jargons}
\label{sec:introduction:notation}
We list some notations and Greek letters that are used in this thesis for reference,
see Table \ref{table:notations}.
The symbols appear mainly in Ch.\,\ref{chap:HMM}.
Some abbreviations are also included in the table.
        \begin{table}[!hbt]
        \center
        \caption{Notations and abbreviations}
        \label{table:notations}
        \begin{tabular}{c p{25em}}
        \hline
        \multicolumn{1}{c}{Notations/Abbreviations}	&	\multicolumn{1}{c}{Meanings}	\\	
        \hline
		$S_t$	&	the latent variable (hidden state) at time $t$	\\	
		$s_t$	&	the value of $S_t$	\\	
		$\bS^{(t)},\bS^{(t+1:T)}$	&	the vector $(S_1,S_2,\dots,S_t)$ and $(S_{t+1},\dots,S_{T})$ 	\\	
		$\bs^{(t)},\bs^{(t+1:T)}$	&	the value vetor of $\bS^{(t)}$ and $\bS^{(t+1:T)}$	\\	
		$X_t$	&	the observed variable at time $t$	\\	
		$x_t$	&	the value of $X_t$	\\	
		$\bX^{(t)},\bX^{(t+1:T)}$	&	the vector $(X_1,X_2,\dots,X_t)$ and $(X_{t+1},\dots,X_{T})$ 	\\	
		$\bx^{(t)},\bx^{(t+1:T)}$	&	the value vetor of $\bX^{(t)}$ and $\bX^{(t+1:T)}$	\\	
		$\delta_i(t)$	&	the probability of state $i$ at time $t$	\\	
		$\bdelta(t)$	&	the probability distribution of states at time $t$	\\	
		$\bGamma$	&	the Markov state transition matrix	\\	
		$\gamma_{ij}$	&	the element at the $i^{th}$ row and $j^{th}$ column of $\bGamma$	\\	
		$\bpi$	&	the initial distribution of the states	\\	
		$\balpha_t$	&	the row vector of forward probabilities at time $t$	\\	
		$\bbeta_t$	&	the row vector of backward probabilities at time $t$	\\	
		$\xi_{ti}$	&	the joint probability of state $i$ at time $t$	\\	
		$\prob{\cdot}$	&	the probability function	\\	
		$E(\cdot)$	&	the expectation function	\\	
		w.r.t.	&	with respect to	\\	
		iff	&	if and only if	\\	
		PDF & 	probability density function	\\
		PMF & 	probability mass function	\\
		CDF & 	cumulative distribution function	\\
		Ch.	&	Chapter	\\	
		Sec.	&	Section (including subsections and subsubsections)	\\	
		Def.	&	Definition	\\
		Thm.	&	Theorem	\\	
		Prop.	&	Proposition	\\	
		Cor.	&	Corollary	\\
		Illus.	&	Illustration	\\	
		Eq.	&	Equation	\\	
        \hline
        \end{tabular}
        \end{table}

There are some terms that may be linguistically ambiguous 
for readers unfamiliar with the financial industry.
Here we specify some terms that are used multiple times but may represent different things.

By \textbf{state transition matrix}, we refer to the state transition probability matrix
of a Markov process.

By \textbf{conditional distribution}, we refer to the state-dependent distribution of 
the observed variable, depending on the context,
or simply refer to a conditional probability distribution.

By \textbf{bear}, we mean the so-called bearish market or bear market for most of the time,
and sometimes it refers to the name of the corresponding state 
in K-Means clustering of the hidden Markov model.
Similar cases are for \textbf{intermediate} and \textbf{bull}.

By \textbf{rebounce}, we mean the reversion of a market trend.
It can either refer to the rally after a period of fall,
or the plunge following the rise.

By \textbf{prediction}, we refer to the stock return prediction and also
the forecast step in a Markov chain of HMM.